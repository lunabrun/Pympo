\section{Approach}
\label{sec:approach}

%Methodological and conceptual approach of the thesis and ideas/plans of implementation.
At first, a simplified analytic model of the bone-\ac{FDP} system will be developed. This model will be used as a base case for calculation of fatigue damage using \ac{SN} fatigue curves from the literature and for verification of a \ac{FE} model for a simplified geometry of the bone-\ac{FDP} system. The \ac{FE} model will be created using the structural module of the software \emph{Ansys Workbench 2020}.

Once the initial FE model is verified, successively more complex and realistic models will be created considering different factors of influence, as for example: preload \citep{joern2016}, type of support  \citep{rand2016}, type of force application \citep{rand2017}, etc. This initial model will at this stage consider only single static loads.

Ideally, at this point of the project, mechanical tests will be performed to validate experimentally the \ac{FE} model. Alternatively, already existing tests (e.g. see \citep{kohorst2007, herzog2009, schneemann2006}) can be reproduced via simulation with the created model to expedite the work and avoid the repetition of tests already performed within the research group. 
The objective of these tests are not only to validate the \ac{FE} model, but also to obtain the parameters for \ac{SN} fatigue curves for the ceramic material (\ac{Y-TZP}) used.

With a verified and validated FE model, the next step is its further development to include structural fatigue effects. This will be realized via an automation script to be written with the programming language \emph{Python} \citep{python2009}. It will, based on a input list of load-cycles, automatically drive the simulation of all load cases and calculate the cumulative damage on pre-defined points based on the calibrated S-N curves.

As a further development of the \ac{FE} model, a damage routine for the jaw bone will be implemented (also in \emph{Python}), to include the effect of bone loss due to peri-implantitis.
This damage routine will progressively reduce the base support of the implant via elimination of mesh elements representing the bone around the base of the implant (i.e., reduction of the rotational rigidity).
Also this \ac{FE} model shall be validate via experiments.
These experiments can be similar to the ones performed previously, with the main difference being in the reduction of the base support during the long-term cyclic loading.
