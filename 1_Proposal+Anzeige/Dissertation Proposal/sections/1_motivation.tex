\section{Motivation}
\label{sec:motivation}

%Description of the scientific context (including the classification in the literature, projects, ...) and of the concrete embedding.

\ac{FDP} are used for the replacement of lost teeth.
They exist in different configurations (crowns, implant, bridges, etc.) and materials (mainly titanium and ceramics) \citep{byrne2014}.
Their use has been increasing in the general population, but specially among the elderly \citep{sato2020}.

A typical ceramic used for such applications is zirconium dioxide (also known as zirconia), specially in its \ac{Y-TZP} form \citep{byrne2014, shemtovyona2019}.

Due to the typical functional load expected for its application (mainly non-uniform varying load due, among others, to mastication), fatigue is one of the main concerns for the long term reliability of \ac{FDP} \citep{shemtovyona2016}.

Another issue regarding the reliability of dental implants is the bone loss due to peri-implantitis, as it can lead to a reduced structural support of the implant \citep{byrne2014}.

Therefore, it is of interest to investigate the interaction between these two mechanisms of failure for \ac{FDP}, as both effects (fatigue and peri-implantitis) are of the long-term cumulative damage type.
It is expected that a progressing peri-implantitis will reduce the support rigidity of a dental implant (either as a single implant or as support for a bridge), leading to higher stresses for the same functional loads, which in turn can lead to a premature fatigue failure.

The main hypothesis for the proposed dissertation is that there is a quantifiable interaction mechanism between these two effects (damage due to fatigue and peri-implantitis). 
This would allow to answer when and under which circumstances the mechanical failure of the \ac{FDP} will occur.

This topic is considered a proper fit to the research group of Prof. Dr. Stiesch, as an extensive range of work performed under her orientation, both in  experimental \citep{kohorst2007, herzog2009} as also in simulation-based \citep{dittmer2007, joern2016, rand2017} studies, already cover the structural strength of \ac{FDP} (both dental implants as also bridges) under different conditions.