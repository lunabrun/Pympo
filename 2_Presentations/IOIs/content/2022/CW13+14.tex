% ------------------------------------------------------------
\section{Calendar Week}
% ------------------------------------------------------------
% --------------------------------------------------- Slide --
\subsection{CW 13+14}
% ------------------------------------------------------------
\begin{frame}
  \frametitle{Review CW 13+14}
	\begin{itemize}
		\item Started transferring code from standard APDL to Python API for Ansys using the pyAnsys project (see \url{https://mapdldocs.pyansys.com/}, official open-source project supported by Ansys, no extra license needed). Reason: allow to have a better readable and maintainable code in Python instead of APDL, but still have acess to all APDL functions. See a first source code example in attachment to email.  \textcolor{yellow}{In-Work}
		\item First positive results for new script using alternative algorithms. \textcolor{yellow}{In-Work}
	\end{itemize}
\end{frame}

% ------------------------------------------------------------
% --------------------------------------------------- Slide --
\subsection{CW 15}
% ------------------------------------------------------------
% ------------------------------------------------------------
\begin{frame}
  \frametitle{Outlook CW 15}
	\begin{itemize}
		\item Reproduce bone remodeling algorithm of the other two papers mentioned in the first slide (Mullender1994 and Lian2010), which can be considered as evolution of the Weinans1992 algorithm. This should allow to get rid of the checker-board issue shown in the results of the previous slide.
		\item Further development of own pyAnsys/APDL script for bone remodeling. Consider a simplified geometry of bone and implant with a single material.
	\end{itemize}
\end{frame}
% --------------------------------------------------- Slide --

