% ------------------------------------------------------------
\section{Calendar Week}
% ------------------------------------------------------------
% --------------------------------------------------- Slide --
\subsection{CW 39}
% ------------------------------------------------------------
\begin{frame}
  \frametitle{Review CW 39}
	\begin{itemize}
		\item Fatigue calculation studies for first test case. \textcolor{green}{Done}
		\begin{itemize}
			\item Used at first SN (i.e., stress-life method) curve for structural steel, as it is readly available in Ansys.
			\item Very simple geometry used. A round cantilever beam with the diameter of a typical implant (4 mm).
			\item "Buccolingual" load of 50N considered (same as in study reproducing the Kitamura2004 results).
			\item Number of "masticatory" cycles per year = 250 000 cycles
			\item Initial damage of bone = 0mm (i.e., nominal position)
			\item Rate of bone "damage", represented by longer beam = 1mm after 5 years, for 20 years in total.
		\end{itemize}
	\end{itemize}
	\begin{center}
	\begin{tabular}{|c|c|c|} 
 		\hline
 		Year & Damage of bone(mm) & Number of cycles\\ 
 		0  & 0 & 1.25 million \\ 
 		5  & 1 & 1.25 million \\ 
 		10 & 2 & 1.25 million \\ 
 		15 & 3 & 1.25 million \\ 
		\hline
	\end{tabular}
	\end{center}
\end{frame}

% ------------------------------------------------------------
% --------------------------------------------------- Slide --
\subsection{CW 40}
% ------------------------------------------------------------
% ------------------------------------------------------------
\begin{frame}
  \frametitle{Outlook CW 40}
	\begin{itemize}
		\item Compare solution of first Finite Element simple model with analytical solution calculated "by hand".
		\item Continue to develop fatigue calculation for test case (even simpler geometry at first), but now for titanium material. 
	\end{itemize}
\end{frame}
% --------------------------------------------------- Slide --

